% Options for packages loaded elsewhere
\PassOptionsToPackage{unicode}{hyperref}
\PassOptionsToPackage{hyphens}{url}
%
\documentclass[
]{article}
\usepackage{lmodern}
\usepackage{amssymb,amsmath}
\usepackage{ifxetex,ifluatex}
\ifnum 0\ifxetex 1\fi\ifluatex 1\fi=0 % if pdftex
  \usepackage[T1]{fontenc}
  \usepackage[utf8]{inputenc}
  \usepackage{textcomp} % provide euro and other symbols
\else % if luatex or xetex
  \usepackage{unicode-math}
  \defaultfontfeatures{Scale=MatchLowercase}
  \defaultfontfeatures[\rmfamily]{Ligatures=TeX,Scale=1}
\fi
% Use upquote if available, for straight quotes in verbatim environments
\IfFileExists{upquote.sty}{\usepackage{upquote}}{}
\IfFileExists{microtype.sty}{% use microtype if available
  \usepackage[]{microtype}
  \UseMicrotypeSet[protrusion]{basicmath} % disable protrusion for tt fonts
}{}
\makeatletter
\@ifundefined{KOMAClassName}{% if non-KOMA class
  \IfFileExists{parskip.sty}{%
    \usepackage{parskip}
  }{% else
    \setlength{\parindent}{0pt}
    \setlength{\parskip}{6pt plus 2pt minus 1pt}}
}{% if KOMA class
  \KOMAoptions{parskip=half}}
\makeatother
\usepackage{xcolor}
\IfFileExists{xurl.sty}{\usepackage{xurl}}{} % add URL line breaks if available
\IfFileExists{bookmark.sty}{\usepackage{bookmark}}{\usepackage{hyperref}}
\hypersetup{
  hidelinks,
  pdfcreator={LaTeX via pandoc}}
\urlstyle{same} % disable monospaced font for URLs
\setlength{\emergencystretch}{3em} % prevent overfull lines
\providecommand{\tightlist}{%
  \setlength{\itemsep}{0pt}\setlength{\parskip}{0pt}}
\setcounter{secnumdepth}{-\maxdimen} % remove section numbering

\date{}

\begin{document}

\begin{center}\rule{0.5\linewidth}{\linethickness}\end{center}

\hypertarget{layout-post-title-using-matlabs-fill-function-with-time-series-data-date-2018-04-04-categories-programming-matlab-visualization}{%
\subsection{layout: post title: "Using MATLAB's fill function with time
series data" date: 2018-04-04 categories: programming MATLAB
visualization}\label{layout-post-title-using-matlabs-fill-function-with-time-series-data-date-2018-04-04-categories-programming-matlab-visualization}}

Recently, I was working on plotting some time series for a model with 6
variables. I wanted to visualize the solutions for this process as a
mean model solution +/- one standard deviation, with a semi-transparent
fill between the standard deviations and had a little trouble
remembering how to do this.

MATLAB's documentation tells you that the \texttt{fill} function makes
polygons, with the vertices specified by the x and y values you supply.

\{\% highlight matlab \%\}

\begin{verbatim}
x = [0 0 1 1] % x values of vertices
y = [0 1 1 0] % y values of vertices 
fill(x,y,'k')
\end{verbatim}

\{\% endhighlight matlab \%\}

\{\% highlight matlab \%\} x = {[}0 0 1 1{]} \% x values of vertices y =
{[}0 1 0 1{]} \% y values of vertices fill(x,y,'k') \{\% endhighlight
matlab \%\}

To draw time series data with those crafty standard deviation shadings,
we can draw a polygon that goes "out and back". The x values should
monotonically increase, and then turn around and monotonically decrease
in the same way. The trick in MATLAB is to use the \texttt{flip}
function, which reverses a vector.

\{\% highlight matlab \%\}

\begin{quote}
\begin{quote}
\begin{quote}
t = {[}1 2 3 5 8 10{]}; full\_tvec = {[}t flip(t){]}
\end{quote}
\end{quote}
\end{quote}

full\_tvec =

\begin{verbatim}
 1     2     3     5     8    10    10     8     5     3     2     1
\end{verbatim}

\{\% endhighlight matlab \%\} Very out and back.

Suppose we have a data matrix where the rows are different time points
and the columns are different observations of the variable at those time
points.\\
From this data, I can compute the mean and standard deviation at each
time point. Finally, the \texttt{fill} function can be used to plot the
time series, where the top vertices are the mean +1 standard deviation
and the bottom vertices are the mean -1 standard deviation.

\{\% highlight matlab \%\}

\% Generate data t = 1:100; data = randn(100,5); data\_mean =
mean(data,2)'; stdv = std(data,{[}{]},2)';

\% Plotting hold on h = fill({[}t flip(t){]},{[}data\_mean+stdv
flip(data\_mean-stdv){]},'k'); set(h,'facealpha',.5)
plot(t,data\_mean,'k') xlabel('Time','Fontsize',20)

\{\% endhighlight matlab \%\}

\end{document}
